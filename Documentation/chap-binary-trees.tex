\chapter{Binary trees}
\label{chap-binary-trees}

\Defclass {node}

This class is the base class for all node types in binary trees.

\Defgeneric {left} {node}

Given a node, return the left child of that node.  The value returned
by this function may be either an instance of (a subclass of) the
class \texttt{node} or \texttt{nil} if the left child of \textit{node}
is the empty tree.

\Defgeneric {(setf left)} {new-left node}

Given a tree \texttt{new-left} and a node \texttt{node}, set the left
child of \texttt{node} to \texttt{new-left}.  The argument
\texttt{new-left} can be either a node or \texttt{nil}.

\Defgeneric {right} {node}

Given a node, return the right child of that node.  The value returned
by this function may be either an instance of (a subclass of) the
class \texttt{node} or \texttt{nil} if the right child of
\textit{node} is the empty tree.

\Defgeneric {(setf right)} {new-right node}

Given a tree \texttt{new-right} and a node \texttt{node}, set the
right child of \texttt{node} to \texttt{new-right}.  The argument
\texttt{new-right} can be either a node or \texttt{nil}.

An \texttt{:after} method is provided with both parameters specialized
to\\ \texttt{node-with-parent}.  This method calls \texttt{(setf
  parent)} with \texttt{node} and \texttt{new-right} as arguments.

\Defclass {simple-node}

This class is a subclass of the class \texttt{node}.  It does not add
any slots, and is only meant for client code to be able to distinguish
between a node with a parent and a node without one.

\Defclass {node-with-parent}

This class is a subclass of the class \texttt{node}.  It adds a slot
for storing a reference to the parent node in a binary tree.
